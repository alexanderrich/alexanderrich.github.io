\documentclass[12pt]{my_cv}
\usepackage[margin=3cm]{geometry}
\usepackage{array}
\newcolumntype{L}{p{0.2\textwidth}}
\newcolumntype{R}{p{0.74\textwidth}}
\def\ind{\hangindent=1 true cm\hangafter=1 \noindent}

\begin{document}

\noindent \textbf{\sffamily\Large Alexander S. Rich}\vspace{1em}\\
Department of Psychology\\
New York University\\
6 Washington Place\\
 New York, NY 10003\\
email: asr443@nyu.edu\\

\section{Education}
\begin{tabular}{L R}
2013--present&New York University\\
&Psychology: Cognition and Perception\\
&Ph.D. expected 2018\\
&Advisor: Dr.\ Todd Gureckis \vspace{1em}\\ 
2009--2013&Williams College\\
&B.A. Mathematics, \emph{Magna Cum Laude}, Phi Beta Kappa\\
&Concentration with Honors in Cognitive Science\\
&Advisor: Dr.\ Safa Zaki\\
\end{tabular}

\section{Honors and Fellowships}
\begin{tabular}{L R}
2015, 2014&Robert J. Glushko and Pamela Samuelson Foundation Student Travel Grant for attendance at the annual meeting of the Cognitive Science Society\\[1ex]
2015&Honorable Mention, National Science Foundation Graduate Research Fellowship\\[1ex]
2013--2018&MacCracken Fellowship, New York University\\
\end{tabular}

\section{Peer Reviewed Journal Articles}

\ind Gureckis, T.M., Martin, J., McDonnell, J., \textbf{Rich, A.S.}, Markant,
D., Coenen, A., Halpern, D., Hamrick, J.B., Chan, P. (2015, in press) psiTurk: and
open-source framework for conducting replicable behavioral experiments online.
\emph{Behavior Research Methods}

\section{Refereed Conference Proceedings}

\ind \textbf{Rich, A.S.} and Gureckis, T.M. (2015). ``The Attentional Learning Trap and How to Avoid It" in P. Bello, M. Guarini, M. McShane, \& B. Scassellati (Eds.) \emph{Proceedings of the 37th Annual Conference of the Cognitive Science Society.} Austin, TX: Cognitive Science Society.	

\ind \textbf{Rich, A.S.} and Gureckis, T.M. (2014). ``The value of approaching bad things" in P. Bello, M. Guarini, M. McShane, \& B. Scassellati (Eds.) \emph{Proceedings of the 36th Annual Conference of the Cognitive Science Society.} Austin, TX: Cognitive Science Society.	

\section{Professional Presentations}
\ind Markant, D., Chan, P., Coenen, A., Martin, J., McDonnell, J., \textbf{Rich, A.} and Gureckis, T. M. (November 2014). ``psiTurk: An open-source framework for conducting reproducible behavioral experiments online.'' Poster presented at the 2014 meeting of the Society for Computers in Psychology, Long Beach, CA, USA.

\ind \textbf{Rich, A.S.} and Gureckis, T.M. (August 2014). ``The value of approaching bad things'' Talk presented at the $36^{th}$ Annual Meeting of the Cognitive Science Society, Quebec City, Canada.

\ind de Leeuw, J.R., Coenen, A., Markant, D., Martin, J.B., McDonnell, J.V., \textbf{Rich, A.S.}, Gureckis, T.M. (2014) ``Workshop: Online Experiments Using jsPsych, psiTurk, and Amazon Mechanical Turk'' in P. Bello, M. Guarini, M. McShane, \& B. Scassellati (Eds.) Proceedings of the 36th Annual Conference of the Cognitive Science Society. Austin, TX: Cognitive Science Society.	

\section{Research Experience}
\begin{tabular}{L R}
2013--present& Graduate Research Assistant\\
&Advisor: Dr.\ Todd Gureckis, New York University\\[1ex]
2012--2013& Research Fellow\\
&Advisor: Dr.\ Safa Zaki, Williams College\\[1ex]
2011&Research Assistant\\
&PI: Dr.\ Frank Keil, Yale University\\
\end{tabular}

\section{Professional Service}

Ad-hoc journal reviewing: \emph{Annual Meeting of the Cognitive Science Society,
Cognitive Science}

\section{Teaching}
\begin{tabular}{L R}
Fall 2015 & Teaching Assistant, \emph{Laboratory in Human Cognition} (undergraduate, NYU)\\ [0.5ex]
Fall 2014&Teaching Assistant, \emph{Mathematical Tools for Neural and Cognitive Science} (graduate, NYU)\\ [0.5ex]
Spring 2013&Teaching Assistant, \emph{Data Structures and Advanced Programming} (undergraduate, Williams)\\[0.5ex]
Fall 2012&Teaching Assistant, \emph{Introduction to Computer Science} (undergraduate, Williams)\\[0.5ex]
Spring 2012&Teaching Assistant, \emph{Linear Algebra} (undergraduate, Williams)\\[0.5ex]
Fall 2010&Teaching Assistant, \emph{Calculus II} (undergraduate, Williams)\\[0.5ex]
\end{tabular}

\end{document}